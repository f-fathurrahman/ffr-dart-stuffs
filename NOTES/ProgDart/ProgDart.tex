\documentclass[a4paper,11pt]{article} % screen setting

\usepackage[a4paper]{geometry}
\geometry{verbose,tmargin=1.5cm,bmargin=1.5cm,lmargin=1.5cm,rmargin=1.5cm}

\setlength{\parskip}{\smallskipamount}
\setlength{\parindent}{0pt}

%\usepackage{fontspec}
\usepackage[libertine]{newtxmath}
\usepackage[no-math]{fontspec}
\setmainfont{Linux Libertine O}
\setmonofont{DejaVu Sans Mono}

\usepackage{hyperref}
\usepackage{url}
\usepackage{xcolor}

% DARKMODE
%\pagecolor[rgb]{0,0,0} %black
%\color[rgb]{0.8,0.8,0.8} %grey

\usepackage{amsmath}
\usepackage{amssymb}

\usepackage{graphicx}
\usepackage{float}

\usepackage{minted}

\newminted{dart}{breaklines,fontsize=\footnotesize}
\newminted{bash}{breaklines,fontsize=\footnotesize}
\newminted{text}{breaklines,fontsize=\footnotesize}

\newcommand{\txtinline}[1]{\mintinline[breaklines,fontsize=\footnotesize]{text}{#1}}
\newcommand{\dartinline}[1]{\mintinline[breaklines,fontsize=\footnotesize]{python}{#1}}

\newmintedfile[pythonfile]{python}{breaklines,fontsize=\footnotesize}

\definecolor{mintedbg}{rgb}{0.90,0.90,0.90}
\usepackage{mdframed}
\BeforeBeginEnvironment{minted}{
    \begin{mdframed}[backgroundcolor=mintedbg,%
        topline=false,bottomline=false,%
        leftline=false,rightline=false]
}
\AfterEndEnvironment{minted}{\end{mdframed}}


\usepackage{setspace}

\onehalfspacing

\usepackage{appendix}


\newcommand{\highlighteq}[1]{\colorbox{blue!25}{$\displaystyle#1$}}
\newcommand{\highlight}[1]{\colorbox{red!25}{#1}}


\begin{document}


\title{Dasar Pemrograman Bahasa Dart}
\author{Fadjar Fathurrahman}
\date{}
\maketitle

\section{Tujuan}

\begin{itemize}
\item Mengenal dan mengunakan tools yang tersedia pada Dart SDK
\item Memahami konsep dasar bahasa pemrograman Dart
\item Mampu membuat program sederhana pada console dengan menggunakan Dart
\end{itemize}

\section{Persiapan}

Pastikan bahwa program-program atau pengaturan berikut ini berjalan
dengan baik pada komputer yang Anda gunakan.

\subsection{(Windows only) cmder}
Pastikan bahwa \txtinline{cmder} dengan Git for Windows sudah terpasang
dan berjalan dengan baik. Jika belum silakan pasang terlebih dahulu.
\txtinline{cmder} dapat diunduh pada tautan berikut:

{\footnotesize
\url{https://github.com/cmderdev/cmder/releases/download/v1.3.16/cmder.zip}
}

\subsection{Flutter SDK}

Pastikan bahwa Flutter SDK sudah terpasang dengan baik. Kita akan mengasumsikan
bahwa yang digunakan adalah Flutter channel beta yang memungkinkan kita untuk
membuat aplikasi Web dan juga Android atau iOS.

Flutter SDK (channel beta) dapat diunduh pada tautan berikut ini.
Cari Beta Channel dan pilih yang file yang sesuai dengan sistem operasi
yang Anda gunakan.

{\footnotesize
\url{https://flutter.dev/docs/development/tools/sdk/releases}
}

Flutter SDK sudah memiliki Dart SDK sehingga kita tidak perlu menginstal
Dart SDK secara terpisah.

\subsection{Setting PATH}

Pastikan bahwa Anda sudah dapat mengakses perintah \txtinline{flutter}
dan \txtinline{dart} pada \txtinline{cmder}. Jika belum, Anda dapat menambahkannya
dengan cara mengedit file
\begin{textcode}
C:\Users\jendela\cmder\config\user_profile.cmd
\end{textcode}
Silakan menyesuaikan bagian
\begin{textcode}
\txtline{C:\Users\jendela\cmder}
\end{textcode}
menjadi lokasi di mana \txtinline{cmder} berada.

Pada file \txtinline{user_profile.cmd}, silakan tambahkan baris berikut
\begin{textcode}
set "PATH=C:\Users\jendela\flutter\bin;%PATH%"
set "PATH=C:\Users\jendela\flutter\bin\cache\dart-sdk\bin
\end{textcode}


\subsection{Visual Studio Code atau text editor lain}

Visual Studio Code (VSCode) dapat diunduh pada tautan berikut.

{\footnotesize
\url{https://code.visualstudio.com/download}
}

Pasang ektensi Flutter pada VSCode. Instruksi terkait dapat ditemukan pada
tautan-tautan berikut:

{\footnotesize
\url{https://flutter.dev/docs/get-started/editor?tab=vscode}
}

{\footnotesize
\url{https://flutter.dev/docs/development/tools/vs-code}
}


\section{Struktur kode program Dart}

Bahasa pemrograman Dart memiliki banyak kesamaan dengan bahasa
pemrograman C/C++.

Program Dart memiliki satu entry point, yaitu fungsi \dartinline{main}.
Fungsi ini biasanya diberikan classifier atau tipe \dartinline{void}.
\begin{dartcode}
void main() {
  // kode program
}
\end{dartcode}

Suatu program Dart dapat terdiri dari beberapa fungsi yang dapat didefinisikan
pada di satu file yang sama.


Dart mendukung tiga tipe komentar.
\begin{dartcode}
// Inline comment

/*
Block comment.
Bisa mencakup lebih dari satu baris
*/

/// Untuk Dokumentasi
///
/// Berikut ini adalah dokumentasi kode.
\end{dartcode}

Tipe data dasar pada Dart mirip dengan tipe data yang ada pada C/C++.
Contoh: \dartinline{int}, \dartinline{double} dan \dartinline{String}.
Perbedaannya adalah semua data tersebut pada Dart merupakan suatu object.
Kita akan mempelajari mengenai object-oriented programming pada praktikum
selanjutnya.


\section{Latihan Program}

Program Dart ditulis dalam file dengan ekstensi \txtinline{.dart}.

Untuk program pertama, silakan buka VSCode dan ketikkan kode berikut ini.
Beri nama file tersebut dengan nama \txtinline{hello.dart} atau nama yang lain.
Usahakan nama tersebut tidak mengandung karakter spasi atau karakter tidak umum
yang lain. Ingat lokasi di mana Anda menyimpan file ini. Sangat disarankan untuk menyimpan
file ini pada satu direktori atau folder tersendiri.

\begin{dartcode}
void main() {
  print('Hello Dart');
}
\end{dartcode}

Buka \txtinline{cmder}, kemudian pindah ke direktori atau folder di mana Anda
menyimpan file \txtinline{hello.dart} sebelumnya.
Gunakan perintah \txtinline{cd} untuk berpindah dari satu direktori ke
direktori lainnya.
Jika hal ini sudah dilakukan, ketikkan perintah berikut ini pada
\txtinline{cmder}. Apa yang Anda amati?

\begin{textcode}
dart hello.dart
\end{textcode}

Jalan juga perintah berikut.
\begin{textcode}
dart2native hello.dart
\end{textcode}
Perintah ini akan menghasilkan file baru. Bisakah Anda menemukan file tersebut?

Perhatikan bawah, pada sistem operasi Windows Anda mungkin perlu menambahkan ekstensi
\txtinline{.exe} sehingga perintah sebelumnya menjadi
\txtinline{dart.exe} dan \txtinline{dart2native.exe}

\subsection{Tugas 1}

Perhatikan kode berikut. Ketikkan pada file yang berbeda.
\begin{dartcode}
void main() {
  sayHello();
}

void sayHello() {
  print('Hello Dart');
}
\end{dartcode}

Dengan menggunakan pengetahuan bahasa C yang sudah Anda pelajari, apakah
output dari program tersebut?

Gunakan perintah \txtinline{dart} atau \txtinline{dart2native}.


\subsection{Tugas 2}

Perhatikan kode berikut. Ketikkan pada file yang berbeda.
\begin{dartcode}
void main() {
  String nama = 'Darto';
  sayHello(nama);
}
  
void sayHello(String name) {
  print('Hello $name');
}
\end{dartcode}

Apakah keluaran dari program ini?


\subsection{Tugas 3}

Perhatikan kode berikut. Ketikkan pada file yang berbeda.
\begin{dartcode}
void main() {
  String Nama = 'Darto';
  String Name = 'Parto';
  sayHello(nama);
  sayHello(name);
}
  
void sayHello(String nama) {
  print('Hello $name');
}
\end{dartcode}

Apakah keluaran dari program ini? Apakah ada error yang dilaporkan?
Apakah Anda dapat memperbaiki program ini?


\subsection{Tugas 4}

Perhatikan kode program berikut.
\begin{dartcode}
void main() {
  List<String> daftar_nama = [
    'Darto',
    'Parto',
    'Sule',
    'Andre',
    'Nunung'
  ];

  for(var nama in daftar_nama) {
    sayHello(nama);
  }
}

void sayHello(String name) {
  print('Hello $name');
}
\end{dartcode}

Apakah keluaran dari program tersebut?

Program ini mengenalkan penggunaan tipe data (atau kelas) \dartinline{List} yang mirip
dengan array pada C/C++.


\subsection{Tugas 5}

Perhatikan program berikut.
\begin{dartcode}
import 'dart:io';

void main() {
  stdout.writeln('Say hello to my bro');
  String input = stdin.readLineSync();
  sayHello(input);
}

void sayHello(String name) {
  print('Hello $name, my bro. Nice to meet you');
}  
\end{dartcode}

Apakah keluaran dari program tersebut?

Catatan: Anda perlu menginputkan nama Anda atau teks lain, diikuti dengan tekan
tombol Enter pada keyboard.

Bandingkan program tersebut dengan program berikut ini.
Apakah ada perbedaan yang Anda amati?
\begin{dartcode}
import 'dart:io';

void main() {
  stdout.writeln('Say hello to my bro');
  stdout.write('Your name bro? ');
  String input = stdin.readLineSync();
  sayHello(input);
}

void sayHello(String name) {
  print('Hello $name, my bro. Nice to meet you');
}  
\end{dartcode}

Dari pengamatan di atas apakah Anda dapat memberikan perbedaan atau persamaan
antara:
\begin{itemize}
\item \dartinline{print}
\item \dartinline{stdout.write}
\item \dartinline{stdout.writeln} 
\end{itemize}


\section{Tugas tambahan 1}

Apakah yang dimaksud dengan variabel PATH dan environment variable
pada Windows / OSX / Linux ?
Bagaimana cara mengubah variabel ini?
Bacaan berikut ini dapat dijadikan acuan (salah satunya).

{\footnotesize
\url{https://superuser.com/questions/284342/what-are-path-and-other-environment-variables-and-how-can-i-set-or-use-them}

\url{https://en.wikipedia.org/wiki/PATH_(variable)}

\url{https://www.computerhope.com/issues/ch000549.htm}

\url{https://superuser.com/questions/502358/easier-way-to-change-environment-variables-in-windows-8}
}


\section{Tugas tambahan 2}

Sintaks untuk melakukan percabangan/kondisional dan looping pada Dart juga sangat mirip
dengan C/C++. Buatlah masing-masing 4 program sederhana untuk demonstrasi
percabangan dan looping pada Dart.

Referensi:

{\footnotesize
\url{https://dart.dev/guides/language/language-tour}
}





\bibliographystyle{unsrt}
\bibliography{BIBLIO}

\end{document}

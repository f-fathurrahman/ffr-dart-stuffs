\documentclass[a4paper,11pt]{article} % screen setting

\usepackage[a4paper]{geometry}
\geometry{verbose,tmargin=1.5cm,bmargin=1.5cm,lmargin=1.5cm,rmargin=1.5cm}

\setlength{\parskip}{\smallskipamount}
\setlength{\parindent}{0pt}

%\usepackage{fontspec}
\usepackage[libertine]{newtxmath}
\usepackage[no-math]{fontspec}
\setmainfont{Linux Libertine O}
\setmonofont{DejaVu Sans Mono}

\usepackage{hyperref}
\usepackage{url}
\usepackage{xcolor}

% DARKMODE
%\pagecolor[rgb]{0,0,0} %black
%\color[rgb]{0.8,0.8,0.8} %grey

\usepackage{amsmath}
\usepackage{amssymb}

\usepackage{graphicx}
\usepackage{float}

\usepackage{minted}

\newminted{dart}{breaklines,fontsize=\footnotesize}
\newminted{bash}{breaklines,fontsize=\footnotesize}
\newminted{text}{breaklines,fontsize=\footnotesize}

\newcommand{\txtinline}[1]{\mintinline[breaklines,fontsize=\footnotesize]{text}{#1}}
\newcommand{\dartinline}[1]{\mintinline[breaklines,fontsize=\footnotesize]{python}{#1}}

\newmintedfile[pythonfile]{python}{breaklines,fontsize=\footnotesize}

\definecolor{mintedbg}{rgb}{0.90,0.90,0.90}
\usepackage{mdframed}
\BeforeBeginEnvironment{minted}{
    \begin{mdframed}[backgroundcolor=mintedbg,%
        topline=false,bottomline=false,%
        leftline=false,rightline=false]
}
\AfterEndEnvironment{minted}{\end{mdframed}}


\usepackage{setspace}

\onehalfspacing

\usepackage{appendix}


\newcommand{\highlighteq}[1]{\colorbox{blue!25}{$\displaystyle#1$}}
\newcommand{\highlight}[1]{\colorbox{red!25}{#1}}


\begin{document}


\title{Pengenalan Pemrograman Berbasis Objek dengan Dart}
\author{Fadjar Fathurrahman}
\date{}
\maketitle

\section{Tujuan}

\begin{itemize}
\item Mengenal konsep dasar pemrograman berbasis objek dan implementasinya dalam Dart.
\item Mampu membuat program sederhana berdasarkan pemrograman berbasis objek
\end{itemize}

\section{Pemrograman berbasis objek}

Pemrograman berbasis dan berorientasi objek (\textit{object oriented programming},
sering disingkat OOP)
adalah suatu paradigma pemrograman yang berpusat pada konsep "objek"
yang secara garis besar terdiri dari:
\begin{itemize}
\item data, \textit{fields}, atribut, atau \textit{properties}
\item prosedur, fungsi, atau metode
\end{itemize}

"Objek" dapat berupa abstraksi dari suatu objek fisik (benda) atau nonfisik yang
dapat ditemukan pada dunia nyata, seperti manusia, mahasiswa, hewan, tumbuhan,
bangunan, universitas, sekolah, sensor, pabrik, dan sebagainya.

Terdapat banyak bahasa pemrograman yang mendukung OOP seperti Java, C++, Python,
C\#, dan Dart. Sebagian besar bahasa pemrograman tersebut mengimplementasikan
OOP dengan menggunakan \textit{class}, yang dapat dianggap sebagai cetak biru
(blue print) dari sebuah objek. Dalam hal ini, objek dikatakan sebagai \textit{instance}
dari sebuah \textit{class}.

Kumpulan dari data dan prosedur dari sebuah objek yang didefinisikan pada suatu
\textit{class} sering juga disebut dengan \textit{class members}.

\subsection{Contoh sederhana}

Sebagai contoh, kita ingin mendeskripsikan suatu objek kucing yang akan
diimplementasikan dalam suatu kelas dengan nama \txtinline{Kucing}.
Kelas \txtinline{Kucing} akan memiliki field berupa nama, warna, dan umur.

\begin{dartcode}
class Kucing {
  String nama;
  String warna;
  int umur;
}
\end{dartcode}

Untuk membuat suatu instance dari \txtinline{Kucing}, kita dapat mengunakan
kode sebagai berikut:
\begin{dartcode}
Kucing kucingku = Kucing();
\end{dartcode}
Kemudian, kita dapat mengakses member dari \txtinline{kucingku} sebagai
berikut:
\begin{dartcode}
kucingku.nama = 'Ronaldo';
kucingku.warna = 'putih';
kucingku.umur = 2;
\end{dartcode}

Sekarang \txtinline{kucingku} dapat dianggap suatu variabel dan dapat digunakan
dalam kode Dart yang lain. Misalnya kita dapat mendeklarasikan suatu fungsi
dengan argumen bertipe \txtinline{Kucing}.
\begin{dartcode}
void sayHello(Kucing k) {
  print('Miaw, namaku adalah ${k.nama}');
}
\end{dartcode}


\subsection{Konstruktor dan member function}
Selain dengan cara manual, kita dapat menginisialisasi suatu objek dengan
menggunakan metode atau fungsi khusus yang dinamai dengan konstruktor.
Dart menyediakan sintaks khusus untuk konstruktor. Program diatas dapat
ditulis sebagai berikut.
\begin{dartcode}
class Kucing {
  String nama;
  String warna;
  int umur;

  Kucing(this.nama, this.warna, this.umur); // konstruktor
}
  
void sayHello(Kucing k) {
  print('Miaw, namaku adalah ${k.nama}');
}

void main() {
  Kucing kucingku = Kucing('Ronaldo', 'hitam', 1);
  sayHello(kucingku);
}
\end{dartcode}

Sintaks yang digunakan pada konstruktor di atas merupakan versi ringkas
dari konstruktor berikut.
\begin{dartcode}
Kucing(String nama, String warna, int umur) {
  this.nama = nama;
  this.warna = warna;
  this.umur = umur;
}
\end{dartcode}

Kita juga dapat membuat fungsi \txtinline{sayHello} menjadi fungsi member.
\begin{dartcode}
class Kucing {
  String nama;
  String warna;
  int umur;

  Kucing(this.nama, this.warna, this.umur);

  void sayHello() {
    print('Miaw, namaku adalah ${this.nama}');
  }

}
  
void main() {
  Kucing kucingku = Kucing('Ronaldo', 'hitam', 1);
  kucingku.sayHello();
}
\end{dartcode}

Kelas \txtinline{Kucing} dapat digunakan untuk membuat objek atau instance
yang lain. Misalnya:
\begin{dartcode}
void main() {
  Kucing kucingku = Kucing('Ronaldo', 'hitam', 1);
  kucingku.sayHello();

  Kucing kucingmu = Kucing('Messi', 'putih', 2);
  kucingmu.sayHello();
}
\end{dartcode}

\subsection{Contoh lain}

\begin{dartcode}
class Ayam {
  String nama;
  String warna;
  int umur;
  String gender;

  Ayam(String nama, String warna, int umur, String gender) {
    this.nama = nama;
    this.warna = warna;
    this.umur = umur;
    this.gender = gender;
  }

  void sayHello() {
    if(this.gender == 'jantan') {
      print('Kukuruyuk, namaku adalah ${this.nama}');
    }
    else if(this.gender == 'betina') {
      print('Petok petok, namaku adalah ${this.nama}');
    }
    else {
      print('Kukuruyuk petok, namaku adalah ${this.nama}');
    }
  }

}
  
void main() {
  Ayam ayam1 = Ayam('Andre', 'hitam', 1, 'jantan');
  ayam1.sayHello();

  Ayam ayam2 = Ayam('Nunung', 'hitam', 1, 'betina');
  ayam2.sayHello();

  Ayam ayam3 = Ayam('XXX', 'hitam', 1, 'lainnya');
  ayam3.sayHello();
}
\end{dartcode}

\subsection{Parameter fungsi opsional dengan kata kunci}

\begin{dartcode}
void myFunc({int a: 2, int b: 3, String name: 'Sule'}) {
  print('a = $a');
  print('b = $b');
  print('name = $name');
}
  
void main() {
  myFunc();
  myFunc(a: 123);
  myFunc(name: 'Andre');
}
\end{dartcode}



\subsection{Inheritance}

Pada banyak kasus kita dapat menemukan bahwa terdapat suatu hierarki pada banyak
\textit{class} yang kita buat. Misalnya, kita dapat menggabungkan kelas
\txtinline{Cat} dan \txtinline{Chicken} ke dalam suatu \textit{super-class}
atau \textit{parent class} \txtinline{Animal}.
Dalam kasus ini dikatakan bahwa \txtinline{Cat} dan \txtinline{Chicken} mewarisi
atau \textit{inherited from} \txtinline{Animal}

\begin{dartcode}
class Animal {
  String name;
  int legCount;
}
  
class Cat extends Animal {
  String makeNoise() {
    print('Miaw');
  }
}
  
class Chicken extends Animal {
  String makeNoise() {
    print('Kukuruyuk');
  }
}
  
void main() {
  Cat cat = Cat();
  cat.name = 'Nora';
  cat.legCount = 4;
  print(cat);
  cat.makeNoise();
  
  Chicken chicken = Chicken();
  chicken.name = 'Kukuk';
  chicken.legCount = 2;
  print(chicken);
  chicken.makeNoise();
}
\end{dartcode}

\section{Tugas}

Buatlah sebuah program untuk mengolah nilai mahasiswa. Nilai mahasiswa terdiri
dari komponen nilai UTS, nilai UAS, dan nilai praktikum. Nilai akhir dihitung dari
hasil pembobotan komponen tiga komponen nilai tersebut (Anda dapat menentukan
sendiri). Nilai akhir juga dapat dikonversi menjadi dalam skala huruf seperti
A, AB, B, BC, C, dan E (Anda juga dapat menentukan aturan konversi yang diperlukan).
Data nilai mahasiswa dibaca melalui suatu file CSV (comma-separated values):

\begin{textcode}
Andre;A19001;90;80;85
Brewok;A19002;60;50;70
Cucun;A19003;70;80;80
Dodo;A19004;70;80;90
.... dan seterusnya
\end{textcode}

Program ini dapat dibuat dengan menggunakan paradigma non-OOP, meskipun demikian
kita akan menggunakan OOP untuk membuat program ini.
Anda perlu mendefinisikan kelas \txtinline{Mahasiswa} beserta membernya untuk
keperluan ini.


\bibliographystyle{unsrt}
\bibliography{BIBLIO}

\end{document}

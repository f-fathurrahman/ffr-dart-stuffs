%\documentclass[a4paper,11pt]{article} % print setting
\documentclass[a4paper,11pt]{article} % screen setting

\usepackage[a4paper]{geometry}
%\geometry{verbose,tmargin=1.5cm,bmargin=1.5cm,lmargin=1.5cm,rmargin=10.0cm}

\setlength{\parskip}{\smallskipamount}
\setlength{\parindent}{0pt}

%\usepackage{cmbright}
%\renewcommand{\familydefault}{\sfdefault}

%\usepackage{fontspec}
\usepackage[libertine]{newtxmath}
\usepackage[no-math]{fontspec}
\setmainfont{Linux Libertine O}
\setmonofont{DejaVu Sans Mono}


\usepackage{hyperref}
\usepackage{url}
\usepackage{xcolor}

% DARKMODE
%\pagecolor[rgb]{0,0,0} %black
%\color[rgb]{0.8,0.8,0.8} %grey

\usepackage{amsmath}
\usepackage{amssymb}

\usepackage{graphicx}
\usepackage{float}

\usepackage{minted}

\newminted{dart}{breaklines,fontsize=\footnotesize}
\newminted{bash}{breaklines,fontsize=\footnotesize}
\newminted{text}{breaklines,fontsize=\footnotesize}

\newcommand{\txtinline}[1]{\mintinline[breaklines,fontsize=\footnotesize]{text}{#1}}
\newcommand{\dartinline}[1]{\mintinline[breaklines,fontsize=\footnotesize]{python}{#1}}

\newmintedfile[pythonfile]{python}{breaklines,fontsize=\footnotesize}

\definecolor{mintedbg}{rgb}{0.90,0.90,0.90}
\usepackage{mdframed}
\BeforeBeginEnvironment{minted}{
    \begin{mdframed}[backgroundcolor=mintedbg,%
        topline=false,bottomline=false,%
        leftline=false,rightline=false]
}
\AfterEndEnvironment{minted}{\end{mdframed}}


\usepackage{setspace}

\onehalfspacing

\usepackage{appendix}


\newcommand{\highlighteq}[1]{\colorbox{blue!25}{$\displaystyle#1$}}
\newcommand{\highlight}[1]{\colorbox{red!25}{#1}}


\begin{document}


\title{Persiapan Praktikum Flutter}
\author{Fadjar Fathurrahman}
\date{}
\maketitle

\section{Tujuan}

\begin{itemize}
\item Praktikan dapat menginstall Flutter SDK pada komputer masing-masing
\item Praktikan dapat menjalankan Flutter pada baris perintah (\textit{command line})
atau terminal
\item Praktikan dapat menjalankan contoh aplikasi Flutter di dekstop (melalui browser)
dan Android/iOS.
\end{itemize}

\section{Pengenalan Flutter}

Flutter adalah suatu kerangka (\textit{framework}) pengembangan antarmuka
pengguna (user interaface) pada aplikasi mobile dan desktop. Flutter dikembangkan
oleh Google dan bersifat open source. Flutter tersedia pada sistem operasi Windows,
Mac OSX dan Linux. Dengan menggunakan Flutter, pengembang (developer) dapat menggunakan
satu kode sumber untuk mengembangkan aplikasi mobile dan desktop.

\section{Instalasi terminal atau console emulator}

Sebagian besar praktikum akan dilakukan dengan menggunakan baris perintah
yang harus diketikkan pada terminal atau console emulator.
Pada Linux dan OSX, sudah tersedia console emulator dengan kemampuan yang baik
sehingga Anda dapat menggunakan terminal bawaan dari operating system.
Untuk Windows, terminal bawaan sistem, yaitu \txtinline{cmd.exe}, tidak memiliki
user interface sebaik pada OSX dan Linux. Praktikan disarankan untuk menggunakan
terminal berbasis PowerShell yang lebih fleksibel. Pilihan lain adalah dengan
menggunakan console emulator lain seperti \txtinline{cmder} yang dapat diunduh
pada {\footnotesize\url{https://cmder.net/}} and pilih full version.
File yang diunduh berupa file zip. Anda dapat mengekstraknya ke folder manapun.
Pada praktikum ini disarankan untuk mengekstraknya ke pada folder user aktif yang
sedang Anda gunakan misalnya:
\txtinline{C:\Users\username\cmder}.

Perhatikan demonstrasi yang akan dilakukan mengenai penggunaan \txtinline{cmder}.

\section{Instalasi Flutter}

Download release beta channel pada laman:


Tambahkan ke path berikut ke environment variable
\txtinline{C:\Users\jendela\flutter\bin}
\txtinline{C:\Users\jendela\flutter\bin\cache\dart-sdk\bin}

Alternatif
Edit file
\begin{textcode}
\txtinline{C:\Users\jendela\cmder\config\user_profile.cmd}
\end{textcode}

\begin{textcode}
set "PATH=C:\Users\jendela\flutter\bin;%PATH%"
set "PATH=C:\Users\jendela\flutter\bin\cache\dart-sdk\bin
\end{textcode}


{\footnotesize\url{https://flutter.dev/docs/development/tools/sdk/releases}}

{\footnotesize\url{https://flutterstudio.app/}}

{\footnotesize\url{https://developer.android.com/studio#downloads}}

Perintah:

\begin{textcode}
flutter config --enable-web
\end{textcode}

Teori

Stateless widgets: komponen UI yang properties-nya tidak pernah berubah.
Mereka tidak memiliki state (keadaan), mereka tidak berubah sendiri melalui
aksi atau perilaku internal. Mereka diubah melalui event eksternal pada parent widgets
yang berada pada widget tree.


Test Code
\begin{dartcode}
class MyApp extends StatelessWidget {
  //...
}
\end{dartcode}



\bibliographystyle{unsrt}
\bibliography{BIBLIO}

\end{document}

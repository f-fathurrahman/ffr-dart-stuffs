\documentclass[a4paper,11pt]{article} % screen setting

\usepackage[a4paper]{geometry}
\geometry{verbose,tmargin=1.5cm,bmargin=1.5cm,lmargin=1.5cm,rmargin=1.5cm}

\setlength{\parskip}{\smallskipamount}
\setlength{\parindent}{0pt}

%\usepackage{fontspec}
\usepackage[libertine]{newtxmath}
\usepackage[no-math]{fontspec}
\setmainfont{Linux Libertine O}
\setmonofont{DejaVu Sans Mono}

\usepackage{hyperref}
\usepackage{url}
\usepackage{xcolor}

% DARKMODE
%\pagecolor[rgb]{0,0,0} %black
%\color[rgb]{0.8,0.8,0.8} %grey

\usepackage{amsmath}
\usepackage{amssymb}

\usepackage{graphicx}
\usepackage{float}

\usepackage{minted}

\newminted{dart}{breaklines,fontsize=\footnotesize}
\newminted{bash}{breaklines,fontsize=\footnotesize}
\newminted{text}{breaklines,fontsize=\footnotesize}

\newcommand{\txtinline}[1]{\mintinline[breaklines,fontsize=\footnotesize]{text}{#1}}
\newcommand{\dartinline}[1]{\mintinline[breaklines,fontsize=\footnotesize]{python}{#1}}

\newmintedfile[pythonfile]{python}{breaklines,fontsize=\footnotesize}

\definecolor{mintedbg}{rgb}{0.90,0.90,0.90}
\usepackage{mdframed}
\BeforeBeginEnvironment{minted}{
    \begin{mdframed}[backgroundcolor=mintedbg,%
        topline=false,bottomline=false,%
        leftline=false,rightline=false]
}
\AfterEndEnvironment{minted}{\end{mdframed}}


%\usepackage{setspace}
%\onehalfspacing

\usepackage{appendix}

\newcommand{\highlighteq}[1]{\colorbox{blue!25}{$\displaystyle#1$}}
\newcommand{\highlight}[1]{\colorbox{red!25}{#1}}

\newcounter{soal}[section]
\newenvironment{soal}[1][]{\refstepcounter{soal}\par\medskip
   \noindent \textbf{Soal~\thesoal. #1} \rmfamily}{\medskip}

\begin{document}


\title{Quiz Dart dan Flutter 1}
\author{}
\date{}
\maketitle

\begin{soal}
Tuliskan path atau lokasi di mana Flutter dan Dart terinstalasi pada komputer Anda:
\begin{itemize}
\item Lokasi file \txtinline{flutter.exe} atau \txtinline{flutter.sh}
\item Lokasi file \txtinline{dart.exe}
\end{itemize}
\end{soal}

\begin{soal}
Jelaskan yang dimaksud dengan object oriented programming dan berikan contoh sederhana
pada bahasa Dart.
\end{soal}

\begin{soal}
Jelaskan yang dimaksud dengan anynymous function pada Dart beserta contohnya.
\end{soal}

\begin{soal}
Buat sebuah kelas sederhana pada Dart dengan spesifikasi berikut.
\begin{itemize}
\item Nama kelas \txtinline{SensorTemperatur}
\item Data/field: nomor seri atau ID, jenis, dan hasil dan waktu pengukuran.
\item Misalkan hasil dan waktu pengukuran berupa array dengan ukuran 100. Anda dapat menggunakan
string untuk waktu pengukuran atau kelas \txtinline{DateTime} pada Dart.
\item Fungsi-fungsi yang harus diimplementasikan:
  \begin{itemize}
  \item konstruktor
  \item fungsi \txtinline{getTemperature} untuk mendapatkan hasil pengukuran.
  Sebagai percobaan Anda dapat menggunakan bilangan acak normal sebagai hasil pengukuran.
  \item fungsi \txtinline{saveData} untuk menyimpan waktu hasil pengukuran
  ke suatu file (misalkan file CSV)
  \end{itemize}
\end{itemize}
\end{soal}

\begin{soal}
Buat aplikasi Flutter dengan spesifikasi sebagai berikut:
\begin{itemize}
\item Judul aplikasi pada AppBar: Developer Info
\item menampilkan foto Anda (close up), bisa berupa pass foto atau foto lain.
\item menampilkan logo Akmet dengan ukuran lebih kecil dari foto.
\item menampilkan nama dan NIM Anda pada dua baris terpisah
\item Ketika foto Anda diklik, maka aplikasi akan memainkan file audio rekaman
sendiri dengan suara Anda.
\end{itemize}
\end{soal}

\bibliographystyle{unsrt}
\bibliography{BIBLIO}

\end{document}
